\documentstyle[12pt,a4]{article}
\begin{document}

%Deliverables
%Evaluation detail

\vfil

\centerline{\Large Computer Science Project Proposal}
\vspace{0.4in}
\centerline{\Large Building HDL abstractions from an RTL 
model in Haskell}
\vspace{0.4in}
\centerline{\large Alex Horsman, Homerton College}
\vspace{0.3in}
\centerline{\large 11th October 2012}

\vfil


\noindent
{\bf Project Supervisor:} Dr. Simon Moore
\vspace{0.2in}

\noindent
{\bf Director of Studies:} Dr. Bogdan Roman
\vspace{0.2in}
\noindent
 
\noindent
{\bf Project Overseers:}
Dr. Peter Robinson \& Dr. Robert Watson

\vfil

\section*{Introduction}

Currently, the only widely supported hardware description 
languages are Verilog and VHDL. While these languages are a 
vast improvement on older techniques of manual circuit 
design, as designs become ever more complex, more powerful 
languages will be needed.

Existing alternatives these low level languages usually 
provide a higher level model which is then compiled to 
Verilog.  However in many cases, while the model is 
effective at designing certain kinds of system, it can be 
awkward to use for others.  Additionally, the compiler may 
not be able to make use of microarchitectural tricks that 
would usually be used by a designer at a lower level.

In order to provide multiple powerful abstractions, while 
still allowing microarchitectural freedom where it is 
desired, the aim of this project is to start with a low 
level model, and build higher level models on top of it.  
These higher level models can then be connected together 
through a common interface, such that each component of a 
system can be created in an appropriate abstract model.

Haskell has many properties useful for this goal. Its syntax 
is very flexible, allowing arbitrary binary operators to be 
defined, and providing a notation for monadic operations. In 
addition, its type system allows very powerful abstractions 
and generalisations to be made, while maintaining type 
safety.

\section*{Starting Point}

There are a few of relevant libraries in the Haskell package 
database which I may use in my project. In particular there 
are some libraries for working with VHDL and Verilog. It is 
likely I will also make use of other more general libraries 
from this package database as well.
In addition, during the summer, I created a library for 
abstracting connections in Bluespec, and I may use a similar 
design in my language's library.

\section*{Substance and Structure}

The core of the project will consist of three components. A 
representation for RTL circuit models, a generator to 
transform the representation to Verilog code, and a series 
of library functions for working with the representation, 
which should be designed to form an embedded language.

The RTL representation will essentially represent a subset 
of Verilog, indicating what registers exist, what logic 
connects them, and what signal edges trigger an update.  
Other features of Verilog may be included if it turns out 
their reimplementation at a higher level is either less 
efficient, or cumbersome. For example, it may prove useful 
to be able to describe a module hierarchy, to assist the 
Verilog compiler by providing synthesis boundaries. This 
representation will exist as a data type in Haskell.

As the representation will correspond closely to Verilog, 
generating this code will involve a relatively simple 
transformation. Each register is given a name and defined, 
similarly intermediate logical values must be named and 
assigned, and finally register update logic is produced as a 
set of "always" blocks representing the different triggers.  

While it would be possible to use the constructors of the 
internal representation as a language to describe hardware 
directly, this is unlikely to be particularly convenient, 
and so a set of library functions will be written to make it 
easier to build circuit representations.

One key feature of the library will be to provide a common 
interface for communicating between circuits described at 
different levels of abstraction. The suggested interface is 
structural groups of input and output wires, which can be 
constrained and abstracted as appropriate in each higher 
level model. For example, a model based on guarded channels 
could constrain its wire interfaces to have appropriate 
guard signals, and expose these as channels internally.

Once the library is created, the project will, as an 
extension, explore various higher level abstractions built 
on top of this base.  For example, one simple model which 
covers many use cases, is single clock synchronous circuits, 
where all register updates are triggered by the same clock 
transition.

\section*{Evaluation}

During the creation of the system, a series of hardware 
examples will be created. This will be used to test the 
functionality of the system, and also to compare with 
existing implementations using other languages, such as 
SystemVerilog and Bluespec.

The main intention of the system is that the code needed to 
create the hardware will be simpler, so this will be the 
main aspect of the comparison. This is of course, somewhat 
subjective, so in addition to a qualitative comparison, a 
few metrics will be compared, such as lines of code.

It is of course also important that any simplicity in 
creation does not come at a significant cost of performance 
in the final design. In order to test this, comparisons will 
also be made of logic elements used, and maximum frequencies 
of the designs produced in the different languages.

\section*{Success Criterion}

\begin{enumerate}

\item
An internal representation of RTL circuit description must 
be designed and implemented.

\item
A program must be written to transform this representation 
into valid Verilog source code.

\item
A set of functions must be written, which provide an 
embedded language for creating RTL circuit descriptions, in 
the internal representation.

\item
Simple hardware examples should be created in the language 
to demonstrate the project works.

\item
Hardware implemented using the language should be compared 
with implementations in other languages, in terms of code 
length as well as logic elements used.

\end{enumerate}

\section*{Plan of Work}

\begin{enumerate}

\item
{\bf 19th October - 2nd November: }
Read papers on Lava, and other related languages. Research 
Verilog synthesis semantics. Set up computers with 
appropriate simulation and synthesis software, and 
familiarise myself with its usage.

\item
{\bf 2nd November - 16th November: }
Design and implement the underlying RTL representation.

\item
{\bf 16th November - 1st December: }
Create the Verilog generator, and test with some very simple 
hardware examples.

\item
{\bf 1st December - 15th December:}
Implement a simple set of functions for building RTL 
circuits, and use them to create some slightly more complex 
examples.

\item
{\bf 15th December - 31st December: }
Implement an abstraction for groups of signals.

\item
{\bf 31st December - 18th January:}
Write progress report. At this stage, the project should be 
capable of describing simple RTL systems, and producing 
correct Verilog to implement them.

\item
{\bf 18th January - 25th January: }
Prepare progress presentation.

\item
{\bf 25th January - 22nd February: }
Implement a simpler interface for creating synchronous 
systems. Explore other extensions.

\item
{\bf 22nd February - 22nd March: }
Write first draft of dissertation.

\item
{\bf 22nd March - 5th April:}
Break to focus on revision, and await feedback.

\item
{\bf 5th April - 21st April: }
Any further work as suggested. Revise and print 
dissertation.

\end{enumerate}

\section*{Resources Required}

I will be using my own computers for the majority of the 
project work. For backup, I will keep my project in a git 
repository, and regularly push it to a number of locations, 
including Github, Dropbox and my external hard drive.

\end{document}
